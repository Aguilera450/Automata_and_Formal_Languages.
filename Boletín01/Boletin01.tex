\documentclass{article}

% Symbols
\usepackage{amsfonts, amsthm}
\usepackage{upgreek}
\usepackage{physics}
\usepackage{cancel}
\usepackage{amssymb, latexsym, amsmath}

%Algorithms
\usepackage[ruled,lined,linesnumbered,commentsnumbered]{algorithm2e}

%% Identación
\setlength{\parindent}{0cm}

% Código
\newcommand{\code}[1]{\textcolor{white!25!black}{\texttt{#1}}}
\usepackage{listings}

%AMS
\usepackage{amsthm}
\newtheorem{algo-thm}{Algoritmo}

% Proof
\renewcommand*{\proofname}{\textbf{Demostraci\'on:}}
% Theorem
\newtheorem*{theorem}{Teorema}

% Graphics
\usepackage{graphicx}
\usepackage{pgf}

% Color a letras.
%\usepackage[usenames,dvipsnames,svgnames,table]{xcolor}

% Tikz
\usepackage{tkz-graph}
\usepackage{tikz}
\usetikzlibrary{arrows,automata}
\usepackage{tikz}
\usetikzlibrary{arrows,automata}
%\usetikzlibrary[topaths]

% Def. Dr. César.
\usetikzlibrary{shapes,calc}
\tikzstyle{edge}=[shorten <=2pt, shorten >=2pt, >=stealth, line width=1.1pt]
\tikzstyle{blueE}=[shorten <=2pt, shorten >=2pt, >=stealth, line width=1.5pt, blue]
\tikzstyle{blackV}=[circle, fill=black, minimum size=6pt, inner sep=0pt, outer sep=0pt]
\tikzstyle{blueV}=[circle, fill=blue, draw, minimum size=6pt, line width=0.75pt, inner sep=0pt, outer sep=0pt]
\tikzstyle{redV}=[circle, fill=red, draw, minimum size=6pt, line width=0.75pt, inner sep=0pt, outer sep=0pt]
\tikzstyle{redSV}=[semicircle, fill=red, minimum size=3pt, inner sep=0pt, outer sep=0pt, rotate=225]
\tikzstyle{blueSV}=[semicircle, fill=blue, minimum size=3pt, inner sep=0pt, outer sep=0pt, rotate=225]
\tikzstyle{blackSV}=[semicircle, fill=black, minimum size=3pt, inner sep=0pt, outer sep=0pt, rotate=225]
\tikzstyle{vertex}=[circle, draw, minimum size=6pt, line width=0.75pt, inner sep=0pt, outer sep=0pt]

% Margins
\addtolength{\voffset}{-1.5cm}
\addtolength{\hoffset}{-1.5cm}
\addtolength{\textwidth}{3cm}
\addtolength{\textheight}{3cm}

%Header-Footer
\usepackage{fancyhdr}
\renewcommand{\headrulewidth}{1pt}

\newcommand{\set}[1]{
  \left\{ #1 \right\}
}

%\pagenumbering{gobble} -- Este comando
%                       -- quita el número de página.
\footskip = 50pt
\renewcommand{\headrulewidth}{1pt}

\pagestyle{fancyplain}

\begin{document}
\title{UNIVERSIDAD AUT\'ONOMA DE M\'EXICO\\ Facultad de Ciencias}
\author{Autor: Adri\'an Aguilera Moreno}
\date{}
\maketitle
\begin{center}
  \includegraphics[scale=0.20]{../Imagen/Portada.jpg}\\[0.4cm]
  \Large
  \bf{Aut\'omatas y Lenguajes Formales}
  \\

  \normalsize
  \begin{center}
    \fbox{
      \begin{minipage}[b][1\height]%
        [t]{0.867\textwidth}
        La propuesta de estos boletines fue hecha por:
        \begin{itemize}
        \item Dr. Favio E. Miranda Parea.
        \item Dra. Lourdes González Huesca.
        \item Mtra. A. Liliana Reyes Cabello.
        \end{itemize}
    \end{minipage}}
  \end{center}
\end{center}
\newpage
\fancyhead[r]{ Aut\'omatas y Lenguajes Formales 2022-2}
%%%%%%%%%%%%%%%%%%%%%%%%%%%%%%%%%%%%%%%%%%%%%%%%%%%%%
\section*{\LARGE{Boletín 1}}
\begin{enumerate}
  %%%%%%%%%%%%%%%%%%%%%%%%%%%% Ejercicio 01  %%%%%%%%%%%%%%%%%%%%%%%%%%%%
\item Sea $w = babbab$ una cadena sobre el alfabeto $\Sigma = \{a, b\}$.
  Describa los conjuntos de \textit{todos} los prefijos y sufijos de $w$.
  ¿Cuáles son propios?
  
  $\triangledown$ \textbf{Solución:}
  
  \textbf{Prefijos:} $\{babbab, babba, babb, bab, ba, b, \lambda\}$.
  
  \textbf{Sufijos:} $\{babbab, abbab, bbab, bab, ab, b, \lambda\}$.
  
  \textbf{Prefijos propios:} $\{babba, babb, bab, ba, b, \lambda\}$.

  \textbf{Sufijos propios:} $\{abbab, bbab, bab, ab, b, \lambda\}$.
  
  \hfill $\lhd$
  %%%%%%%%%%%%%%%%%%%%%%%%%%%% Ejercicio 02  %%%%%%%%%%%%%%%%%%%%%%%%%%%%
\item Demostrar las propiedades de concatenación de cadenas usando inducción:
  \newcommand{\localtextbulletone}{\textcolor{gray}{\raisebox{.45ex}{\rule{.6ex}{.6ex}}}}
  \renewcommand{\labelitemi}{\localtextbulletone}
  \begin{itemize}
    %%\renewcommand{\labelitemi}{} 
  \item Asociatividad: $(uv)w = u(vw)$.
  \item Identidad: $v\lambda = \lambda v = v$.
  \item Longitud: $|vw| = |v| + |w|$.
  \end{itemize}
  \renewcommand\qedsymbol{QED}
  \begin{proof} Consideremos las definiciones recursivas de cadena
    concatenada por la izquierda. Así, analicemos 3 posibles casos:
    \begin{enumerate}
    \item Asociatividad.
      Sea $\Sigma$ un alfabeto, $\forall_{w_1, w_2, w_3} \in \Sigma^{*};$ y $a \in \Sigma:$
      
      Para este inciso haremos inducción sobre la estructura de las cadenas. Si $w_1 = \lambda$,
      entonces
      \begin{eqnarray*}
        w_1 \cdot  (w_2 \cdot w_3) &=& \lambda \cdot (w_2 \cdot w_3)
        \hspace*{2cm} \text{Sabemos que } w_1 = \lambda.\\
        &=& (w_2 w_3)
        \hspace{2.7cm} \text{Identidad en cadenas.}\\
        &=& (w_2 w_3) \cdot \lambda
        \hspace*{2.25cm} \text{Identidad en cadenas.}\\
        &=& (w_2 \cdot w_3) \cdot \lambda
        \hspace*{1.5cm} \text{Definición de concatenación.}
      \end{eqnarray*}
      \begin{center}
        \fbox{
          \begin{minipage}[b][1\height]%
            [t]{0.867\textwidth}
            Obs\'ervese que ``$\cdot$" es usado para indicar la concatenación, cuando se
            omite podemos trabajar esa concatenación como una cadena. En adelante se omite
            esta observación y se emplea de manera indistinta.
        \end{minipage}}
      \end{center}
      Así, supongamos sin pérdida de generalidad que, para alguna cadena $w \in \Sigma^{*}$
      se cumple que $w\cdot (w_{2} \cdot w_{3}) = (w \cdot w_{2}) \cdot w_{3}$, entonces
      \begin{eqnarray*}
        a \cdot \left(w \cdot (w_{2} \cdot w_{3}) \right) &=&  aw \cdot (w_{2} \cdot w_{3})
        \hspace*{1cm} \text{Concatenación de un símbolo y una cadena.}\\
        &=& aw \cdot w_2 w_3
        \hspace*{2cm} \text{Resultado de concatenar $2$ cadenas.}\\
        &=& aw w_2 w_3
        \hspace*{2.3cm} \text{Resultado de concatenar $2$ cadenas.}\\
        &=& aw w_2 \cdot w_3
        \hspace*{2.1cm} \text{Concatenación respecto a un sufijo.}\\
        &=& (a \cdot ww_2) \cdot w_3
        \hspace*{1.55cm} \text{Concatenación respecto a un prefijo.}\\
        &=& (a \cdot (w \cdot w_2)) \cdot w_3
        \hspace*{1.7cm} \text{Definición de concatenación.}
      \end{eqnarray*}
      \hspace*{4.5cm} $\therefore \hspace*{0.5cm} (uv)w = u(vw)$
    \item Identidad.
      Propongamos a $\lambda$ como el neutro para la concatenación de cadenas.
    \item Longitud.
      Sea $\Sigma$ un alfabeto, $\forall_{w_1, w_2} \in \Sigma^{*};$ y $a \in \Sigma:$
            
      Para este inciso haremos inducción sobre la estructura de las
      cadenas. Nótese que si $w_{1} = \lambda$, entonces
      \begin{eqnarray*}
        |w_{1} \cdot w_{2}| &=& |\lambda \cdot w_{2}|
        \hspace*{2.3cm} \text{Recordemos que }  w_{1} = \lambda.\\
        &=& |w_{2}|
        \hspace*{2cm} \text{Concatenación con la cadena vac\'ia.}\\
        &=& 0 + |w_{2}|
        \hspace*{2cm} \text{El cero es el neutro aditivo.}\\
        &=& |\lambda| + |w_{2}|
        \hspace*{2.2cm} \text{Por definición: } |\lambda| = 0.\\
        &=& |w_{1}| + |w_{2}|
        \hspace*{2.1cm} \text{Nuevamente: } w_{1} = \lambda.
      \end{eqnarray*}
      Ahora, supongamos sin pérdida de generalidad que, para alguna cadena
      $w \in \Sigma^{*}$ se cumple que $|w \cdot w_{2}| = |w| + |w_{2}|$, luego
      \begin{eqnarray*}
        |(a \cdot w) \cdot w_{2}| &=& |a \cdot (w \cdot w_{2})|
        \hspace*{2.7cm} \text{Asociatividad en la concatenación.} \\
        &=& 1 + |w \cdot w_{2}|
        \hspace*{1cm} \text{Para $u$ cadena y $b$ símbolo, se tiene que } |b \cdot u| = 1 + |u|. \\
        &=& 1 + |w| + |w_{2}|
        \hspace*{2.6cm} \text{Uso de la hipótesis de inducción.}\\
        &=& |a \cdot w| + |w_{2}|
        \hspace*{0.8cm} \text{Para $u$ cadena y $b$ símbolo, se tiene que } |b \cdot u| = 1 + |u|.\\
      \end{eqnarray*}
      \hspace*{4.5cm} $\therefore \hspace*{0.5cm} |vw| = |v| + |w|$
    \end{enumerate}
  \end{proof}
\item Demostrar que dado un alfabeto cualquiera y para cualesquiera cadenas $u, v, w$ en
  la cerradura, se cumplen las siguientes propiedades:
  \begin{enumerate}
  \item $|w| = |w|$.
  \item $\left(w^{R}\right)^{R} = w$.
  \item $(uv)^{R} = v^R u^R$.
  \item Para cada $n \geq 0,\; \left(w^{n}\right)^{R} = \left(w^{n}\right)^{R}$.
  \end{enumerate}
\item Sea $\Sigma = \{a, b\}$, el lenguaje $L$ se define como:
  \begin{enumerate}
  \item[i)] $\lambda \in L$.
  \item[ii)] Si $w_1, w_2 \in L$, entonces $aw_1bw_2$ y $bw_1aw_2$ pertenecen a $L$.
  \item[iii)] Son todas las cadenas en $L$.
  \end{enumerate}
\item Suponer que $L_1 \geq \{a,b\}^{*}$ esta definido por:
  \begin{enumerate}
  \item[i)] $\lambda \in L_1$.
  \item[ii)]  para cada $w \in L_1$ sucede que $wa$ y $wba$ también pertenecen a $L_1$.
  \end{enumerate}
  Demuestre que para cada $v \in L_1$ las siguientes afirmaciones se cumplen:
  \begin{enumerate}
  \item Cualquier cadena $u \in L_1$ tiene mayor o igual número de $a$'s que de
    $b$'s $\left(\#_{a} \geq \#_{b}\right)$.
  \item Cualquier cadena $u \in L_1$ no contiene la subcadena $bb$.
  \end{enumerate}
\item 
\end{enumerate}
\end{document}
