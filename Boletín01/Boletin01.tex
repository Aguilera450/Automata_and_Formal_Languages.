\documentclass{article}

% Symbols
\usepackage{amsfonts, amsthm}
\usepackage{upgreek}
\usepackage{physics}
\usepackage{cancel}
\usepackage{amssymb, latexsym, amsmath}

%Algorithms
\usepackage[ruled,lined,linesnumbered,commentsnumbered]{algorithm2e}

%% Identación
\setlength{\parindent}{0cm}

% Código
\newcommand{\code}[1]{\textcolor{white!25!black}{\texttt{#1}}}
\usepackage{listings}

%AMS
\usepackage{amsthm}
\newtheorem{algo-thm}{Algoritmo}

% Proof
\renewcommand*{\proofname}{\textbf{Demostraci\'on:}}
% Theorem
\newtheorem*{theorem}{Teorema}

% Graphics
\usepackage{graphicx}
\usepackage{pgf}

% Color a letras.
%\usepackage[usenames,dvipsnames,svgnames,table]{xcolor}

% Tikz
\usepackage{tkz-graph}
\usepackage{tikz}
\usetikzlibrary{arrows,automata}
\usepackage{tikz}
\usetikzlibrary{arrows,automata}
%\usetikzlibrary[topaths]

% Def. Dr. César.
\usetikzlibrary{shapes,calc}
\tikzstyle{edge}=[shorten <=2pt, shorten >=2pt, >=stealth, line width=1.1pt]
\tikzstyle{blueE}=[shorten <=2pt, shorten >=2pt, >=stealth, line width=1.5pt, blue]
\tikzstyle{blackV}=[circle, fill=black, minimum size=6pt, inner sep=0pt, outer sep=0pt]
\tikzstyle{blueV}=[circle, fill=blue, draw, minimum size=6pt, line width=0.75pt, inner sep=0pt, outer sep=0pt]
\tikzstyle{redV}=[circle, fill=red, draw, minimum size=6pt, line width=0.75pt, inner sep=0pt, outer sep=0pt]
\tikzstyle{redSV}=[semicircle, fill=red, minimum size=3pt, inner sep=0pt, outer sep=0pt, rotate=225]
\tikzstyle{blueSV}=[semicircle, fill=blue, minimum size=3pt, inner sep=0pt, outer sep=0pt, rotate=225]
\tikzstyle{blackSV}=[semicircle, fill=black, minimum size=3pt, inner sep=0pt, outer sep=0pt, rotate=225]
\tikzstyle{vertex}=[circle, draw, minimum size=6pt, line width=0.75pt, inner sep=0pt, outer sep=0pt]

% Margins
\addtolength{\voffset}{-1.5cm}
\addtolength{\hoffset}{-1.5cm}
\addtolength{\textwidth}{3cm}
\addtolength{\textheight}{3cm}

%Header-Footer
\usepackage{fancyhdr}
\renewcommand{\headrulewidth}{1pt}

\newcommand{\set}[1]{
  \left\{ #1 \right\}
}

%\pagenumbering{gobble} -- Este comando
%                       -- quita el número de página.
\footskip = 50pt
\renewcommand{\headrulewidth}{1pt}

\pagestyle{fancyplain}

\begin{document}
\title{UNIVERSIDAD AUT\'ONOMA DE M\'EXICO\\ Facultad de Ciencias}
\author{Autor: Adri\'an Aguilera Moreno}
\date{}
\maketitle
\begin{center}
  \includegraphics[scale=0.20]{../Imagen/Portada.jpg}\\[0.4cm]
  \Large
  \bf{Aut\'omatas y Lenguajes Formales}
  \\

  \normalsize
  \begin{center}
    \fbox{
      \begin{minipage}[b][1\height]%
        [t]{0.867\textwidth}
        La propuesta de estos boletines fue hecha por:
        \begin{itemize}
        \item Dr. Favio E. Miranda Parea.
        \item Dra. Lourdes González Huesca.
        \item Mtra. A. Liliana Reyes Cabello.
        \end{itemize}
    \end{minipage}}
  \end{center}
\end{center}
\newpage
\fancyhead[r]{ Aut\'omatas y Lenguajes Formales 2022-2}
%%%%%%%%%%%%%%%%%%%%%%%%%%%%%%%%%%%%%%%%%%%%%%%%%%%%%
\section*{\LARGE{Boletín 1}}
\begin{enumerate}
  %%%%%%%%%%%%%%%%%%%%%%%%%%%% Ejercicio 01  %%%%%%%%%%%%%%%%%%%%%%%%%%%%
\item Sea $w = babbab$ una cadena sobre el alfabeto $\Sigma = \{a, b\}$.
  Describa los conjuntos de \textit{todos} los prefijos y sufijos de $w$.
  ¿Cuáles son propios?
  
  $\triangledown$ \textbf{Solución:}
  
  \textbf{Prefijos:} $\{babbab, babba, babb, bab, ba, b, \lambda\}$.
  
  \textbf{Sufijos:} $\{babbab, abbab, bbab, bab, ab, b, \lambda\}$.
  
  \textbf{Prefijos propios:} $\{babba, babb, bab, ba, b\}$.

  \textbf{Sufijos propios:} $\{abbab, bbab, bab, ab, b\}$.
  
  \hfill $\lhd$
  
\item Demostrar las propiedades de concatenación de cadenas usando inducción:
  \newcommand{\localtextbulletone}{\textcolor{gray}{\raisebox{.45ex}{\rule{.6ex}{.6ex}}}}
  \renewcommand{\labelitemi}{\localtextbulletone}
  \begin{itemize}
    %%\renewcommand{\labelitemi}{} 
  \item Asociatividad: $(uv)w = u(vw)$.
  \item Identidad: $v\lambda = \lambda v = v$.
  \item Lngitud: $|vw| = |v| + |w|$.
  \end{itemize}
  \begin{proof} Para este ejercicio analicemos 3 posibles casos:
    
  \end{proof}
\end{enumerate}
\end{document}
