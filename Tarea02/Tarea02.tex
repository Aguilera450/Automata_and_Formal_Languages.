\documentclass{article}

% Symbols
\usepackage[T1]{fontenc}
\usepackage{upgreek}
\usepackage{physics}
\usepackage{cancel}
\usepackage{amsfonts, amsthm}
\usepackage{amssymb, latexsym, amsmath}

%Algorithms
\usepackage[ruled,lined,linesnumbered,commentsnumbered]{algorithm2e}

%% Identación
\setlength{\parindent}{0cm}

% Código
\newcommand{\code}[1]{\textcolor{white!25!black}{\texttt{#1}}}
\usepackage{listings}

%AMS
\usepackage{amsthm}
\newtheorem{algo-thm}{Algoritmo}

% Proof
\renewcommand*{\proofname}{\textbf{Demostraci\'on:}}
% Theorem
\newtheorem*{theorem}{Teorema}

% Graphics
\usepackage{graphicx}
\usepackage{pgf}

% Color a letras.
%\usepackage[usenames,dvipsnames,svgnames,table]{xcolor}

% Tikz
\usepackage{tkz-graph}
\usepackage{tikz}
\usetikzlibrary{arrows,automata}
\usepackage{tikz}
\usetikzlibrary{arrows,automata}
%\usetikzlibrary[topaths]

% Def. Dr. César.
\usetikzlibrary{shapes,calc}
\tikzstyle{edge}=[shorten <=2pt, shorten >=2pt, >=stealth, line width=1.1pt]
\tikzstyle{blueE}=[shorten <=2pt, shorten >=2pt, >=stealth, line width=1.5pt, blue]
\tikzstyle{blackV}=[circle, fill=black, minimum size=6pt, inner sep=0pt, outer sep=0pt]
\tikzstyle{blueV}=[circle, fill=blue, draw, minimum size=6pt, line width=0.75pt, inner sep=0pt, outer sep=0pt]
\tikzstyle{redV}=[circle, fill=red, draw, minimum size=6pt, line width=0.75pt, inner sep=0pt, outer sep=0pt]
\tikzstyle{redSV}=[semicircle, fill=red, minimum size=3pt, inner sep=0pt, outer sep=0pt, rotate=225]
\tikzstyle{blueSV}=[semicircle, fill=blue, minimum size=3pt, inner sep=0pt, outer sep=0pt, rotate=225]
\tikzstyle{blackSV}=[semicircle, fill=black, minimum size=3pt, inner sep=0pt, outer sep=0pt, rotate=225]
\tikzstyle{vertex}=[circle, draw, minimum size=6pt, line width=0.75pt, inner sep=0pt, outer sep=0pt]

% Margins
\addtolength{\voffset}{-1.5cm}
\addtolength{\hoffset}{-1.5cm}
\addtolength{\textwidth}{3cm}
\addtolength{\textheight}{3cm}

%Header-Footer
\usepackage{fancyhdr}
\renewcommand{\headrulewidth}{1pt}

\newcommand{\set}[1]{
  \left\{ #1 \right\}
}

%\pagenumbering{gobble} -- Este comando
%                       -- quita el número de página.
\footskip = 50pt
\renewcommand{\headrulewidth}{1pt}

\pagestyle{fancyplain}

\begin{document}
\title{UNIVERSIDAD AUT\'ONOMA DE M\'EXICO\\ Facultad de Ciencias}
\author{Autor: Adri\'an Aguilera Moreno}
\date{}
\maketitle
\begin{center}
  \includegraphics[scale=0.20]{../Imagen/Portada.jpg}\\[0.4cm]
  \Large
  \bf{Aut\'omatas y Lenguajes Formales}
  \normalsize
\end{center}
\newpage
\fancyhead[r]{ Aut\'omatas y Lenguajes Formales 2022-2}
%%%%%%%%%%%%%%%%%%%%%%%%%%%%%%%%%%%%%%%%%%%%%%%%%%%%%
\section*{\LARGE{Tarea 2}}
\begin{enumerate}
\item Demuestra que el lenguaje
  \[
  \{\mathbf{a} \in \{a, b, c\}^{*}\; |\; \text{la longitud de $\mathbf{a}$
    es el cuadrado de un natural.}\}
  \]
  no es regular. Usa tanto el teorema del bombeo como el de
  Myhill-Nerode.
  \begin{proof} Antes de iniciar con las demostraciones por los
    teoremas requeridos, analicemos la estructura de las cadenas
    generadas por
    \[
    L(A) = \{\mathbf{a} \in \{a, b, c\}^{*}\; |\; \text{la longitud de $\mathbf{a}$
    es el cuadrado de un natural.}\}
    \]
    Suponiendo que existe el autómata $A \in DFA$ tal que reconoce el lenguaje anterior.
    
    Nótese que $\Sigma = \{a,b,c\}$ y $L(A) = \Sigma^{*}$, entonces $\mathbf{a}$ debe
    ser una cadena finita y supondremos que $L(A)$ es un lenguaje ``muy grande'' pero
    finito.
    
    Además, para nuestras demostraciones nos fijaremos en la longitud de las cadenas
    generadas por nuestro lenguaje y no necesariamente en la estructura de estas
    cadenas, entonces
    \[
    |abababcbacbabc| = |aaaaabbbbbbccc| = |a^{5}b^{6}c^{3}|
    \]
    por tanto, en las demostraciones supondré, sin pérdida de generalidad, que
    \[
    \mathbf{a} = a^{i}b^{j}c^{r}
    \]
    para $i,j,r \in \mathbb{N}$.
    \begin{center}
      \fbox{
        \begin{minipage}[b][1\height]%
          [t]{0.867\textwidth}
          \textbf{Obs.} El que $\mathbf{a}$ tenga la estructura anterior
          no indica que $b$ siempre debe estar en esa posición, incluso
          podriamos precindir de $b$ o de algún término en la cadena, al
          suponer esa estructura solo simplificamos el considerar casos (que
          por ser un conjunto infinito o muy grande, tendriamos muchas
          combinaciones para concatenar elementos del alfabeto).
      \end{minipage}}
    \end{center}
    Para este ejercicio analicemos $2$ versiones en la
    demostración, \textit{i.e.},
    \newcommand{\localtextbulletone}{\textcolor{black}{\raisebox{.45ex}{\rule{.6ex}{.6ex}}}}
    \renewcommand{\labelitemi}{\localtextbulletone}
    \begin{itemize}
    \item \textbf{Demostración por Teorema del Bombeo.} Procedamos por reducción al absurdo,
      supongamos que el lenguaje dado es regular y que $A = \langle Q_{A}, \Sigma, \delta_{A},
      s, \mathbb{F}_{A} \rangle$ es un autómata determinista finito [en caso de no serlo,
        siempre se puede reducir a un $DFA$] con $k$ estados y $L(A)$ como se definió previamente.
      
      Sea $\mathbf{a} = a^{i}b^{j}c^{r}$ una cadena aceptada por nuestro autómata con
      \[
      i + j + r > k \text{ y al menos } j > k.
      \]
      de lo anterior, podemos deducir que durante la lectura de la cadena $\mathbf{a}$, al
      menos en $2$ ocasiones se pasa por algún estado $q$ en $A$ (es claro que $q$ debe ser
      estado de A). Así, para la función $\delta_{A}$ de transición en $A$, tenemos que
      \begin{eqnarray*}
        \delta^{*}(s, a^{i}b^{m}) &=& q\\
        \delta^{*}(q, b^{n}) &=& q\\
        \delta^{*}(q, b^{l}c^{r}) &=& f \in \mathbb{F}_{A}.
      \end{eqnarray*}
      $\forall_{m,n,l} \in \mathbb{N}$. Donde $m + n + l = j$ y en particular $n \neq 0$ [$b^{n}$
        no se colapsa a $\lambda$].
      
      Como $A$ es determinista, tenemos que para todo $p \in \mathbb{N}$ se cumple que
      \[
      \delta^{*}(q, b^{np}) = q
      \]
      como consecuencia, directa se tiene que
      \[
      \delta^{*}(s, a^{i}b^{m}b^{np}b^{l}c^{r}) = f \in \mathbb{F}_{A}.
      \]
      pero $m + np + l \neq j$ cuando $p \neq 1$. Por tanto $ a^{i}b^{m}b^{np}b^{l}c^{r} \in L(A)$,
      lo que implica
      \[
      i + m + np + l + r = t^{2}
      \]
      con $t \in \mathbb{N}$. Sin embargo,
      \[
      i + m + np + l + r = t^{2}\; \not\rightarrow\; i + m + n(p + 1) + l + r = t^{2} + n = x^{2}\; !!!
      \]
      \begin{center}
      \fbox{
        \begin{minipage}[b][1\height]%
          [t]{0.867\textwidth}
          \textbf{Contraejemplo:} Con $i = 6, m = 3, n = 4, l = 2, r = 1$.
          \begin{eqnarray*}
            16 &=& 6 + 3 + 4 + 2 + 1
            \hspace*{4cm} \text{Con } p = 1.\\
            20 &=& 6 + 3 + 4(2) + 2 + 1 = 6 + 3 + 8 + 2 + 1
            \hspace*{0.5cm} \text{Con } p = 2.
          \end{eqnarray*}
          Claramente $20$ no es el cuadrado de un natural.
      \end{minipage}}
    \end{center}      
      lo cual es una contradicción, pues esto debe funcionar para cualquier $p$ natural.
      \[
      \therefore\; \{\mathbf{a} \in \{a, b, c\}^{*}\; |\; \text{la longitud de $\mathbf{a}$
        es el cuadrado de un natural.}\} \text{ \textbf{NO} es regular}.
      \]
    \item \textbf{Demostración por Teorema de Myhill-Nerode.} Por refutación, supongamos que
      
    \end{itemize}

    En todos los casos se obvia que la demostración para $\mathbf{a} = \lambda$ [cadena vacía] es por vacuidad.
  \end{proof}
\end{enumerate}
\end{document}
