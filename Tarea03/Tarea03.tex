\documentclass{article}

% Symbols
\usepackage{amsfonts, amsthm}
\usepackage{upgreek}
\usepackage{physics}
\usepackage{cancel}
\usepackage{amssymb, latexsym, amsmath}

%Algorithms
\usepackage[ruled,lined,linesnumbered,commentsnumbered]{algorithm2e}

%% Identación
\setlength{\parindent}{0cm}

% Código
\newcommand{\code}[1]{\textcolor{white!25!black}{\texttt{#1}}}
\usepackage{listings}

%AMS
\usepackage{amsthm}
\newtheorem{algo-thm}{Algoritmo}

% Proof
\renewcommand*{\proofname}{\textbf{Demostraci\'on:}}
% Theorem
\newtheorem*{theorem}{Teorema}

% Graphics
\usepackage{graphicx}
\usepackage{pgf}

% Color a letras.
%\usepackage[usenames,dvipsnames,svgnames,table]{xcolor}

% Tikz
\usepackage{tkz-graph}
\usepackage{tikz}
\usetikzlibrary{arrows,automata}
\usepackage{tikz}
\usetikzlibrary{arrows,automata}
%\usetikzlibrary[topaths]

% Def. Dr. César.
\usetikzlibrary{shapes,calc}
\tikzstyle{edge}=[shorten <=2pt, shorten >=2pt, >=stealth, line width=1.1pt]
\tikzstyle{blueE}=[shorten <=2pt, shorten >=2pt, >=stealth, line width=1.5pt, blue]
\tikzstyle{blackV}=[circle, fill=black, minimum size=6pt, inner sep=0pt, outer sep=0pt]
\tikzstyle{blueV}=[circle, fill=blue, draw, minimum size=6pt, line width=0.75pt, inner sep=0pt, outer sep=0pt]
\tikzstyle{redV}=[circle, fill=red, draw, minimum size=6pt, line width=0.75pt, inner sep=0pt, outer sep=0pt]
\tikzstyle{redSV}=[semicircle, fill=red, minimum size=3pt, inner sep=0pt, outer sep=0pt, rotate=225]
\tikzstyle{blueSV}=[semicircle, fill=blue, minimum size=3pt, inner sep=0pt, outer sep=0pt, rotate=225]
\tikzstyle{blackSV}=[semicircle, fill=black, minimum size=3pt, inner sep=0pt, outer sep=0pt, rotate=225]
\tikzstyle{vertex}=[circle, draw, minimum size=6pt, line width=0.75pt, inner sep=0pt, outer sep=0pt]

% Margins
\addtolength{\voffset}{-1.5cm}
\addtolength{\hoffset}{-1.5cm}
\addtolength{\textwidth}{3cm}
\addtolength{\textheight}{3cm}

%Header-Footer
\usepackage{fancyhdr}
\renewcommand{\headrulewidth}{1pt}

\newcommand{\set}[1]{
  \left\{ #1 \right\}
}

%\pagenumbering{gobble} -- Este comando
%                       -- quita el número de página.
\footskip = 50pt
\renewcommand{\headrulewidth}{1pt}

\pagestyle{fancyplain}

\begin{document}
\title{UNIVERSIDAD AUT\'ONOMA DE M\'EXICO\\ Facultad de Ciencias}
\author{Autor: Adri\'an Aguilera Moreno}
\date{}
\maketitle
\begin{center}
  \includegraphics[scale=0.20]{../Imagen/Portada.jpg}\\[0.4cm]
  \Large
  \bf{Aut\'omatas y Lenguajes Formales}
  \normalsize
\end{center}
\newpage
\fancyhead[r]{ Aut\'omatas y Lenguajes Formales 2022-2}
%%%%%%%%%%%%%%%%%%%%%%%%%%%%%%%%%%%%%%%%%%%%%%%%%%%%%
\section*{\LARGE{Tarea 3}}
\begin{enumerate}
  %%%%%%%%%%%%%%%%%%%%%%%%%%%%%%%%%%%%%%%%%%%%%%% Ejercicio 01
\item Diseña una máquina de Turing que reconozca el lenguaje
  \[
  \{\alpha\alpha\alpha \big| \alpha \in \in\{a, b\}^*\}
  \]
  A continuación se diseña una máquina de Turing que cumple los
  requerimientos:
  \begin{center}
    \begin{tikzpicture}[estado/.style={circle,draw=black},auto,/tikz/initial text={},
        final/.style={circle,draw=black,double},thick]
      \node [estado,initial, final] (s) [left=100pt]   {$s$};
      \node [estado,final]          (t) [right=100pt]  {$t$};
      \path[->]
      (s) edge [curve to, out=0,in=180,relative]  node {$\vdash(_,_,\rightarrow)\big|(_,_,\rightarrow)|(_,_,\rightarrow)\dashv$} (t)
      (s)  edge [loop below]                      node {$\vdash(a,a,\rightarrow)\big|(a,a,\rightarrow)|(a,a,\rightarrow)\dashv$\;\;
                                                        $\vdash(a,a,\rightarrow)\big|(a,a,\rightarrow)|(a,a,\rightarrow)\dashv$} (s);
    \end{tikzpicture}
  \end{center}
      de esta manera se acepta la división de la cadena original en tres cadenas y la máquina de Turing verifica que estos segmentos
    sean los mismos. \hfill $\square$
  %%%%%%%%%%%%%%%%%%%%%%%%%%%%%%%%%%%%%%%%%%%%%% Ejercicio 02
\item Demuestra (sin utilizar el teorema de Rice) que el conjunto
  \[
  REG = \{\langle M \rangle \big| L(M) \text{ es regular}\}
  \]
  no es recursivamente enumerable y tampoco lo es su complemento.
  \textbf{Sugerencia:} $\emptyset$ y $\Sigma^*$ son regulares.
\end{enumerate}
  
\textbf{\textit{Desmotración:}} Analicemos 2 posibles casos:

1. [$REG$ no es r.e.] Para este caso realicemos la reducción 
\[
\thicksim HP \leq_{m} REG
\]
como consecuencia de lo anterior se tiene que existe una función computable
$\sigma$, tal que si
\[
\langle M \rangle \# \langle \alpha \rangle \in\; \thicksim HP
\Leftrightarrow \sigma\left(\langle M \rangle \# \langle \alpha \rangle\right) \in REG
\]
entonces, definimos a $\sigma$ como
%------------------- Algoritmo para la función computable, 1era reducción.

\begin{algorithm}[H]
  \SetAlgorithmName{}{}%función computable}{}
  \DontPrintSemicolon
  \SetKwData{False}{false}\SetKwData{True}{true}
  \SetKwFunction{New}{new}\SetKwFunction{End}{end}\SetKwFunction{Used}{used}
  \SetKwInOut{Input}{input}\SetKwInOut{Output}{output}

  \KwIn{$M \in TM$ y $\alpha \in \Sigma^*$.}
  \KwOut{Una $M' \in TM$ que simula a $M$.}
  \BlankLine {
    Sea $M'(\beta)$: simulación de $M(\alpha)$\;
    \If{$\beta = 0^n1^n$}{
      termina y acepta\;
    }
    termina y rechaza\;
  }
             {\Return $M'$\;}
             \caption{$\sigma(M, \alpha)$ computable} \label{sigma}%\textnormal{new}
             \DecMargin{1em}
\end{algorithm}

ahora, analicemos dos posibles casos:
\begin{itemize}
\item[$\Rightarrow$)] En este caso, si $M$ no se detiene con la cadena $\alpha$, tenemos entonces
  \begin{eqnarray*}
  \langle M \rangle \# \langle \alpha \rangle \in\; \thicksim HP
  &\Rightarrow& M' \text{ solo puede recibir a la cadena } \alpha = \emptyset.\\
  &\Rightarrow& L(M') = \emptyset \in REG.
  \end{eqnarray*}
  de lo anterior se cumple la ida de la demostración.
\item[$\Leftarrow$)] En este caso particular, procedamos por contrapositiva, esto es
  \begin{eqnarray*}
    \langle M \rangle \# \langle \alpha \rangle \notin\; \thicksim HP
    &\Rightarrow& M \text{ se detiene con } \alpha.\\
    &\Rightarrow& \text{por } \sigma \text{ y la línea 2 del algoritmo, } \beta = 0^n1^n.\\
    &\Rightarrow& L(M') = \{0^n1^n\} \notin REG.
  \end{eqnarray*}
  la última implicación se sigue del \textbf{Lema del Bombeo} [demostración hecha en clase].
  Como tomamos la contrapositiva, esto es equivalente a
  \[ \sigma\left(\langle M \rangle \# \langle \alpha \rangle\right) \in REG
  \Rightarrow \langle M \rangle \# \langle \alpha \rangle \in\; \thicksim HP.\]
\end{itemize}
del análisis anterior podemos concluir que 
\[
\thicksim HP \leq_{m} REG
\]
son reducibles. Como $\thicksim HP$ no es semidecidible, y esto a su vez implica que
no es recursivamente enumerable. Entonces $REG$ no es recursivamente enumerable.

2. [$\thicksim REG$ no es r.e.] Para este caso realicemos la reducción
\[
\thicksim HP \leq_{m}\; \thicksim REG
\]
de esto se sigue que, existe una función computable $\delta$ tal que
\[
\langle M \rangle \# \langle \alpha \rangle \in\; \thicksim HP
\Leftrightarrow \delta\left(\langle M \rangle \# \langle \alpha \rangle\right) \in\; \thicksim REG
\]
entonces, definimos $\delta$ tal que
%------------------- Algoritmo para la función computable, 2da reducción.

\begin{algorithm}[H]
  \SetAlgorithmName{}{}%función computable}{}
  \DontPrintSemicolon
  \SetKwData{False}{false}\SetKwData{True}{true}
  \SetKwFunction{New}{new}\SetKwFunction{End}{end}\SetKwFunction{Used}{used}
  \SetKwInOut{Input}{input}\SetKwInOut{Output}{output}

  \KwIn{$M \in TM$ y $\alpha \in \Sigma^*$.}
  \KwOut{Una $M' \in TM$ que simula a $M$.}
  \BlankLine {
    Sea $M'(\beta)$\;
    \If{$\beta = 0^n1^n$}{
      termina y acepta\;
    }
    simulación de $M(\alpha)$\;
    termina y rechaza\;
  }
             {\Return $M'$\;}
             \caption{$\delta(M, \alpha)$ computable} \label{sigma}%\textnormal{new}
             \DecMargin{1em}
\end{algorithm}
ahora, analicemos dos posibles casos:
\begin{itemize}
\item[$\Rightarrow$)] En este caso, si $M$ no se detiene con la cadena $\alpha$, tenemos entonces
  \begin{eqnarray*}
    \langle M \rangle \# \langle \alpha \rangle \in\; \thicksim HP
    &\Rightarrow& \beta = 0^n1^n.\\
    &\Rightarrow& L(M') = \{0^n1^n\} \notin REG\\
    &\Rightarrow& L(M') \in\; \thicksim REG.
  \end{eqnarray*}
  la penúltima implicación se sigue del \textbf{Lema del Bombeo} [demostración hecha en clase].
\item[$\Leftarrow$)] En este caso particular, procedamos por contrapositiva, esto es
  \begin{eqnarray*}
    \langle M \rangle \# \langle \alpha \rangle \in\;  HP
    &\Rightarrow& M \text{ se detiene con } \alpha.\\
    &\Rightarrow& \text{por } \sigma \text{ y las líneas 5 y 6 del algoritmo aceptan } \Sigma^*.\\
    &\Rightarrow& L(M') \in REG.
  \end{eqnarray*}
  Como tomamos la contrapositiva, esto es equivalente a
  \[ \sigma\left(\langle M \rangle \# \langle \alpha \rangle\right) \in \thicksim REG
  \Rightarrow \langle M \rangle \# \langle \alpha \rangle \in\; \thicksim HP.\]
\end{itemize}
del análisis anterior podemos concluir que 
\[
\thicksim HP \leq_{m} \thicksim REG
\]
son reducibles. Como $\thicksim HP$ no es semidecidible, y esto a su vez implica que
no es recursivamente enumerable. Entonces $\thicksim REG$ no es recursivamente enumerable.
\hfill $\square$

\begin{enumerate}
\item[3.] Demuestra ahora que REG no es ni siquiera recursivamente enumerable
  por medio de uno de los teoremas de Rice.
  \begin{proof} Por el $2^{\text{do}}$ teorema de Rice tenemos que para una
    propiedad no trivial de lenguajes recursivamente enumerables NO monótona, la propiedad
    no es siquiera semidecidible.

    Recordemos que, con refrente a la clasificación de propiedades monótonas,
    la propiedad de ser regular no es monótona. De lo anterior y por el $2^{\text{do}}$
    teorema de Rice, se sigue que REG no es siquiera semidecidible.

    Sabemos que una propiedad es semidecidible si y sólo si es recursivamente enumerable.
    Como REG no es siquiera semidecidible, concluimos por transatividad que
    
    \[ \therefore\;\; REG \text{ no es siquiera recursivamente enumerable.}\]
  \end{proof}
\item[4.] Demuestra que los lenguajes decidibles son cerrados bajo unión y
  concatenación.
  \begin{proof}
    Para este ejercicio, analicemos dos posibles casos:
    \newcommand{\localtextbulletone}{\textcolor{black}{\raisebox{.45ex}{\rule{.6ex}{.6ex}}}}
    \renewcommand{\labelitemi}{\localtextbulletone}
    \begin{itemize}
    \item Cerradura bajo unión:
      
      Sean dos lenguajes cualesquiera $L, L'$ decidibles, y $M, M'$ máquinas de Turing totales
      tales que
      \[L(M) = L \hspace*{0.2cm}\text{ y }\hspace*{0.2cm} L(M') = L'\]
      tomemos un elemento en
      \[x \in L \cup L'\]
      y observemos que $x \in L$ o $x \in L'$, si $x \in L$ entonces esta en un lenguaje decidible y
      por tanto $x$ es decidible. Si $x \in L'$, entonces esta en un lenguaje decidible y nuevamente
      $x$ es decidible.
      \[\therefore\;\; \{\text{Lenguajes decidibles}\} \text{ son cerrados bajo unión.} \]
    \item Sean dos lenguajes cualesquiera $L, L'$ decidibles, y $M, M'$ máquinas de Turing totales
      tales que
      \[L(M) = L \hspace*{0.2cm}\text{ y }\hspace*{0.2cm} L(M') = L'\]
      tomemos
      \[x \in L \hspace*{0.2cm}\text{ y }\hspace*{0.2cm} x' \in L'\]
      y notemos que $xx' \in LL'$, pero $xx'$ es una cadena aceptada por
      $N \in TM$ tal que $N$ es de dos cintas donde una de las cintas es
      la simulación de $M$ y otra cinta es $M'$, entonces basta con recorrer
      y aceptar $x$ con la primer cinta, caso análogo con $x'$ pero con la
      segunda cinta. Luego, $N(xx')$, como $N$ corresponde a dos cintas que
      previamente sabiamos que generaban un lenguaje decidible, entonces
      podemos concluir que $xx'$ son parte de un lenguaje decidible y como
      \[xx' \in LL'\]
      entonces, $LL'$ es decidible.
    \end{itemize}
  \end{proof}
\end{enumerate}
\end{document}
