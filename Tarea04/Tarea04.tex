\documentclass{article}

% Symbols
\usepackage{amsfonts, amsthm}
\usepackage{upgreek}
\usepackage{physics}
\usepackage{cancel}
\usepackage{amssymb, latexsym, amsmath}

%Algorithms
\usepackage[ruled,lined,linesnumbered,commentsnumbered]{algorithm2e}

%% Identación
\setlength{\parindent}{0cm}

% Comentario en bloques:
\iffalse
\fi

% Hipervínculos:
\usepackage{hyperref}

% Código
\newcommand{\code}[1]{\textcolor{white!25!black}{\texttt{#1}}}
\usepackage{listings}

%AMS
\usepackage{amsthm}
\newtheorem{algo-thm}{Algoritmo}

% Proof
\renewcommand*{\proofname}{\textbf{Soluci\'on:}}
% Theorem
\newtheorem*{theorem}{Teorema}

% Graphics
\usepackage{graphicx}
\usepackage{pgf}

% Color a letras.
%\usepackage[usenames,dvipsnames,svgnames,table]{xcolor}

% Tikz
\usepackage{tkz-graph}
\usepackage{tikz}
\usetikzlibrary{arrows,automata}
\usepackage{tikz}
\usetikzlibrary{arrows,automata}
%\usetikzlibrary[topaths]

% Def. Dr. César.
\usetikzlibrary{shapes,calc}
\tikzstyle{edge}=[shorten <=2pt, shorten >=2pt, >=stealth, line width=1.1pt]
\tikzstyle{blueE}=[shorten <=2pt, shorten >=2pt, >=stealth, line width=1.5pt, blue]
\tikzstyle{blackV}=[circle, fill=black, minimum size=6pt, inner sep=0pt, outer sep=0pt]
\tikzstyle{blueV}=[circle, fill=blue, draw, minimum size=6pt, line width=0.75pt, inner sep=0pt, outer sep=0pt]
\tikzstyle{redV}=[circle, fill=red, draw, minimum size=6pt, line width=0.75pt, inner sep=0pt, outer sep=0pt]
\tikzstyle{redSV}=[semicircle, fill=red, minimum size=3pt, inner sep=0pt, outer sep=0pt, rotate=225]
\tikzstyle{blueSV}=[semicircle, fill=blue, minimum size=3pt, inner sep=0pt, outer sep=0pt, rotate=225]
\tikzstyle{blackSV}=[semicircle, fill=black, minimum size=3pt, inner sep=0pt, outer sep=0pt, rotate=225]
\tikzstyle{vertex}=[circle, draw, minimum size=6pt, line width=0.75pt, inner sep=0pt, outer sep=0pt]

% Margins
\addtolength{\voffset}{-1.5cm}
\addtolength{\hoffset}{-1.5cm}
\addtolength{\textwidth}{3cm}
\addtolength{\textheight}{3cm}

%Header-Footer
\usepackage{fancyhdr}
\renewcommand{\headrulewidth}{1pt}

\newcommand{\set}[1]{
  \left\{ #1 \right\}
}

%\pagenumbering{gobble} -- Este comando
%                       -- quita el número de página.
\footskip = 50pt
\renewcommand{\headrulewidth}{1pt}

\pagestyle{fancyplain}

\begin{document}
\title{UNIVERSIDAD NACIONAL AUT\'ONOMA DE M\'EXICO\\ Facultad de Ciencias}
\author{Autor: Adri\'an Aguilera Moreno}
\date{}
\maketitle
\begin{center}
  \includegraphics[scale=0.20]{../Imagen/Portada.jpg}\\[0.4cm]
  \Large
  \bf{Aut\'omatas y Lenguajes Formales}
  \normalsize
\end{center}
\newpage
\fancyhead[r]{ Aut\'omatas y Lenguajes Formales 2022-2}
%%%%%%%%%%%%%%%%%%%%%%%%%%%%%%%%%%%%%%%%%%%%%%%%%%%%%
\section*{\LARGE{Tarea 4}}
Considera la exponenciación:
\begin{eqnarray*}
  \mathrm{n^{0}} &=& \mathrm{1}\\
  \mathrm{n^{m + 1}} &=& \mathrm{n^m \times n}
\end{eqnarray*}
\begin{enumerate}
  %%%~~~~~~~~~~~~~~~~~~~~~~~~~ Problema 01:
\item Transfórmala en una afirmación recursiva $\mu$.
  \begin{proof}
    Antes de definir la función recursiva-$\mu$ requerida, definamos
    una función recursiva-$\mu$ que defina a la siguiente función
    \begin{eqnarray*}
      \mathrm{h(0)} &=& \mathrm{s(0)}\\
      \mathrm{h(s(n))} &=& \mathrm{h(n)}
    \end{eqnarray*}
    como la función recursiva-$\mu$
    \begin{eqnarray*}
      \mathrm{h:}\; \mathbb{N} \rightarrow \mathbb{N} &&\\
      \mathrm{h(\text{cero}())} &=& \mathrm{s \circ cero()}\\
      \mathrm{h(s(n))} &=& \mathrm{\prod_{2}^{2}(n, h(n))}.
    \end{eqnarray*}
    Después de la definición anterior, encontremos una función recursiva-$\mu$
    que cumpla con lo requerido\footnote{Para esto haremos uso de la función
    \code{sum} y \code{mult} vistas en clase.}, esta es
    \begin{center}
      \fbox{
        \begin{minipage}[b][1\height]%
          [t]{0.867\textwidth}
          \textbf{Obs.} la función $\mathrm{g}: \mathbb{N}^3 \rightarrow \mathbb{N}$, se define por
          \[\mathrm{g} = \mathrm{mult \circ \left(\prod_{1}^{3}, \prod_{3}^{3}\right)}\]
          que no resulta tan relevante hacerla explicita en la definición de $\mathrm{expt}$.
      \end{minipage}}
    \end{center}      
    \begin{eqnarray*}
      \mathrm{expt:}\; \mathbb{N}^2 \rightarrow \mathbb{N} &&\\
      \mathrm{expt(n, 0)} &=& \mathrm{h(m)}\\
      \mathrm{expt(n, s(m))} &=& \mathrm{\left(mult \circ \left(\prod_{1}^{3}, \prod_{3}^{3}\right)\right)(n, m, expt(n, m))}.
    \end{eqnarray*}
  \end{proof}
    %%%~~~~~~~~~~~~~~~~~~~~~~~~~ Problema 02:
\item A partir de la respuesta a 1, escríbela de acuerdo con las convenciones
  para el cálculo $\Lambda$.
  \begin{proof}
    Para definir una función-$\lambda$ que transcriba el funcionamiento de la función recursiva-$\mu$ presentada
    con anterioridad necesitamos definir al numeral $1$, esto es
    \begin{eqnarray*}
      \overline{1} &\equiv_{def}& \left[F, \overline{0}\right].
    \end{eqnarray*}
    Ahora, definimos la función-$\lambda$ requerida\footnote{Tomando en cuenta las funciones-$\lambda$ $\mathrm{*, P, Cero}$,
    y $\mathrm{Y}$ vistas en clase.}, esto es
    \begin{eqnarray*}
      \mathrm{expt} &\equiv_{def}& \mathrm{Y\left(\lambda_{f}.\lambda_{nm}(Cero\; m)\;\overline{1}\;(*\; (P\; m)\; n\; (f((P\; m)\; n)))\right)}.
    \end{eqnarray*}
    debe quedar claro que los terminos $n, m$ son la base y el exponente respectivamente.
  \end{proof}
  %%%~~~~~~~~~~~~~~~~~~~~~~~~~ Problema 03:
\item A partir de la respuesta a 1, impleméntala por medio de un programa
  de IMP.
  \newline
  \newline
  \textbf{\textit{Solución:}} Para esta problema, demos las siguientes definiciones recursivas en IMP\footnote{Bajo el supuesto de que $*$ := MULT $\in$ IMP.}
  \begin{eqnarray*}
    \mathrm{HEXPT} &=& \mathrm{X}\; :=\; 1\\
    \mathrm{GEXPT} &=& \mathrm{X\; :=\; X * Z}
  \end{eqnarray*}
  a partir de lo anterior definimos\footnote{Y suponiendo $+$ := SUMA $\in$ IMP.}
  
  \begin{algorithm}[H]
    \SetAlgorithmName{}{}%función computable}{}
    \DontPrintSemicolon
    \SetKwData{False}{false}\SetKwData{True}{true}
    \SetKwFunction{New}{new}\SetKwFunction{End}{end}\SetKwFunction{Used}{used}
    \SetKwInOut{Input}{input}\SetKwInOut{Output}{output}

    \KwIn{$\mathrm{N, M}$ como respectivos en base y potencia.}
    \KwOut{$\mathrm{N^M}$.}
    \BlankLine {
      Y := 0\;
      Z := N\;
      W := M\;
      HEXPT\;
      \While{\textnormal{Y $<$ W}}{
        GEXPT\;
        N := Z\;
        Y := Y + 1\;
      }
    }
               {}
               \caption{EXPT} \label{sigma}%\textnormal{new}
               \DecMargin{1em}
  \end{algorithm}
  \subsubsection*{Comentarios por línea}
  1. Está línea indica nuestro contador.
  \newline
  2. Aquí recibimos la base de la potencia.
  \newline
  3. Recibimos el exponente de la potencia.
  \newline
  4. Llammamos lo equivalente a nuestro caso base (notar que no se
  esta haciendo de manera recursiva).
  \newline
  5. Verificamos la condición del \code{while}.
  \newline
  6. Llamamos a la función auxiliar que realiza la multiplicación de terminos.
  \newline
  7. Como los comandos resultan ser destructivos, recuperamos el valos de N
  (se hace la presición de el por qué no se recupera el valor de M. Pues no
  se utiliza en algún comando destructivo).
  \newline
  8. Se aumenta el contador en 1.
  \newline
  9. Termina el ciclo (esto pasa cuando la condición de validación en 5
  ya no funciona, devuelve F).
  \hfill $\square$
\item Define la semántica del comando
  \begin{center}
    \code{for\; i\; :=\; 1\ to\ n\ do\ P}
  \end{center}
  y utilízalo para definir la recursión primitiva de una manera más directa.
  \newline
  \newline
  \textbf{\textit{Solución:}}
  Para este ejercicio supondremos la regla semántica que define al orden
  ``$<$''\footnote{En general supondremos todas las definiciones dadas en clase,
  pero especificaremos como llamarle para hacerlas más compactas.}, así podemos definir
  \begin{center}                                                             
    \begin{array}{rl}
      &\langle \mathrm{i := 1}, \sigma \rangle \rightarrow \sigma'
      \hspace*{1.8cm} \langle \mathrm{i < n}, \sigma' \rangle \rightarrow F&
      \hline
      &\hspace*{0.4cm} \langle \text{\code{for\; i\; :=\; 1\ to\ n\ do\ P}}, \sigma \rangle
      \rightarrow \sigma&
    \end{array}
  \end{center}
  Nótese que el \code{for} se definió bajo el orden ``$<$'', sin embargo se puede construir
  una versión que admita el orden ``$\leq$'' de la siguiente manera
  \begin{center}                                                             
    \begin{array}{rl}
      &\langle \mathrm{i := 1}, \sigma \rangle \rightarrow \sigma'
      \hspace*{0.5cm} \langle \mathrm{i < n}\; \lor\; \mathrm{i = n},
      \sigma' \rangle \rightarrow F&
      \hline
      &\hspace*{0.4cm} \langle \text{\code{for\; i\; :=\; 1\ to\ n\ do\ P}}, \sigma \rangle
      \rightarrow \sigma&
    \end{array}
  \end{center}
  Para este ejercicio particular tomaré la primer definición y se debe observar que es
  el equivalente a ``\code{h}'' en nuestras funciones recursivas-$\mu$. Basados en esta
  definición, definimos
  \begin{center}                                                             
    \begin{array}{rl}
      &\langle \mathrm{i := 1}, \sigma \rangle \rightarrow \sigma'' \hspace*{0.5cm}
      \langle \mathrm{i < n}, \sigma'' \rangle \rightarrow V
      \hspace*{0.5cm} \langle P, \sigma'' \rangle \rightarrow \sigma'''
      \hspace*{0.5cm} \langle \text{\code{for\; i\; :=\; 1\ to\ n\ do\ P}},
      \sigma''' \rangle \rightarrow \sigma'&
      \hline
      &\hspace*{4cm} \langle \text{\code{for\; i\; :=\; 1\ to\ n\ do\ P}},
      \sigma \rangle \rightarrow \sigma'_{[i := i + 1]}&
    \end{array}
  \end{center}
  Ahora que tenemos la semántica del comando \code{for}, podemos definir la recursión
  primitiva basada en este comando.
  \begin{center}
    \fbox{
      \begin{minipage}[b][1\height]%
        [t]{0.867\textwidth}
        \textbf{Obs.}
        Para la definición de recursión primitiva se supone que existen $P_h$ y $P_g$, pues
        sabemos que existen las funciones $h$ y $g$ como funciones recursivas-$\mu$ y
        por la demostración de la equivalencia entre funciones recursivas-$\mu$ y SOE,
        podemos concluir que sus homólogos [$P_h$ y $P_g$] en SOE existen.
    \end{minipage}}
  \end{center}
  de la lámina\footnote{Otros modelos computacionales.} 30/32 tenemos que
  \begin{eqnarray*}
    \mathrm{\sigma'(X_0)} &=& \mathrm{h(\sigma(X))}
    \hspace*{3cm} \text{donde } \langle P_h, \sigma \rangle \rightarrow \sigma'\\
    \mathrm{\sigma'(X_0)} &=& \mathrm{g(\sigma(Z), \sigma(X), \sigma(X'))}
    \hspace*{1.15cm} \text{donde } \langle P_g, \sigma \rangle \rightarrow \sigma'
  \end{eqnarray*}
  así, definimos el siguiente programa
\end{enumerate}
\begin{algorithm}[H]
  \SetAlgorithmName{}{}%función computable}{}
  \DontPrintSemicolon
  \SetKwData{False}{false}\SetKwData{True}{true}
  \SetKwFunction{New}{new}\SetKwFunction{End}{end}\SetKwFunction{Used}{used}
  %\SetKwInOut{Input}{input}\SetKwInOut{Output}{output}

  %\KwIn{$\mathrm{N, M}$ como respectivos en base y potencia.}
  %\KwOut{$\mathrm{N^M}$.}
  \BlankLine {
    W := Z\;
    Y := X\;
    $\mathrm{P_h}$\;
    \For{\textnormal{i := 1 \textbf{to}} $\mathrm{W}$}{
      Z := i - 1\;
      X := Y\;
      $\mathrm{X'}$ := $\mathrm{X_0}$\;
      $\mathrm{P_g}$\;
    }
  }
             {}
             \caption{Recursión Primitiva.} \label{sigma}%\textnormal{new}
             \DecMargin{1em}
\end{algorithm}
Obsérvese que este programa recorre a $\mathrm{i \in \{1, \dotsm, W - 1\}}$, y $\mathrm{Z}$ va tomando valores desde
\[\mathrm{0 \leq Z \leq W - 2}\]
es por esto que se debe aceptar un valor para Z (que eventualmente se guardará en W) del valor requerido
más uno, esto es, si se quisiera Z. Entonces llamamos al programa con $\mathrm{Z + 1}$, \textit{i.e.},
\[\mathrm{W := Z + 1}\]
será el tope. Así, habrá Z\footnote{W después de la primera iteración, pues aquí se conserva el valor inicial de Z.}
iteraciones\footnote{En el while de la lámina 30/32 justamente se itera de 0 a W - 1 veces, esto es W iteraciones.}.
Caso análogo si utilizaramos la definición alternativa para el for [la que considera a la igualdad].
\hfill $\square$
\end{document}
